\documentclass{beamer}
\title{Résolution de jeux de sûreté joués sur graphes}
\author{Huylenbroeck Florent}
\institute{UMONS}
\date{25 novembre 2022}
\begin{document}

\frame{\titlepage}

\begin{frame}
\frametitle{Sommaire}
\tableofcontents
\end{frame}

\section{Jeux joués sur graphes}
\begin{frame}
\frametitle{Jeux joués sur graphes}
\end{frame}

\section{Cas fini}
\begin{frame}
\frametitle{Cas fini} %exemple %terminer par => attracteur
\end{frame}

\subsection{Attracteurs}
\begin{frame}
\frametitle{Attracteur : définition}
\end{frame}

\begin{frame}
\frametitle{Attracteur : calcul}
\end{frame}

\subsection{Ensemble gagnant}
\begin{frame}
\frametitle{Dualité jeux d'atteignabilité et sûreté}
\end{frame}

\begin{frame}
\frametitle{Exemple}
\end{frame}

\section{Cas infini}
\begin{frame}
\frametitle{Cas infini} %exemple %meilleur def de W %3 étapes
\end{frame}

\subsection{Représentation rationnelle du problème}
\begin{frame}
\frametitle{Automates et transducteurs}
\end{frame}

\begin{frame}
\frametitle{Représentation rationnelle}
\end{frame}

\subsection{Apprentissage}
\begin{frame}
\frametitle{Apprentissage}
\end{frame}

\begin{frame}
\frametitle{Apprentissage : enseignant} % exemple vérif une condition
\end{frame}

\begin{frame}
\frametitle{Apprentissage : élève} % SAT, formule boléenne
\end{frame}

\subsection{Représentation boléenne du problème}
\begin{frame}
\frametitle{Formules boléennes} % exemple d'une formule
\end{frame}

\subsection{Ensemble gagnant}
\begin{frame}
\frametitle{Traduction d'un modèle en un ensemble gagnant}
\end{frame}

\section{Conclusion}
\begin{frame}
\frametitle{Conclusion}
\end{frame}

\section{Questions}
\begin{frame}
\frametitle{Questions}
\end{frame}
\end{document}