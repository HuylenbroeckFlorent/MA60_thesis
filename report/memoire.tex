\documentclass[12pt,a4paper,oneside, titlepage]{report}

\usepackage{times}
\usepackage[frenchb]{babel}
\usepackage{hyperref} 
\usepackage[utf8]{inputenc}
%\usepackage[T1]{fontenc}
\usepackage{amsmath} %%
%\usepackage{amsfonts}
%\usepackage{amscd}
%\usepackage{amstext}
\usepackage{amssymb} %%
%\usepackage{bar}
\usepackage{color}
%\usepackage{mathrsfs}
\usepackage{graphicx}
%\usepackage{calligra}
%\usepackage{amsthm}
%\usepackage{multirow}
%\usepackage{tabularx}
%\usepackage{layout}
%\pagestyle{headings}
\usepackage{fancyhdr}
\usepackage{pdfpages} %
\usepackage{algorithm} %
\usepackage{algpseudocode} %
\usepackage{makecell} %
\pagestyle{fancy}

%\setlength{\textheight}{630pt}
%\setlength{\footskip}{30pt}
\newtheorem{defi}{D\'efinition}[section]
\newtheorem{note}{Note}[section]
\newtheorem{proprietet}{Propri\'et\'e}[section]
\newtheorem{exemple}{Exemple}[section]
\newtheorem{corollaire}{Corollaire}[section]
\newtheorem{rem}{Remarque}[section]
\newtheorem{thm}{Th\'eor\`eme}[section]
\newtheorem{illustration}{Illustration}[section]
\newenvironment{demonstration}{\begin{proof}[\textnormal{\textbf{Preuve.}}]}{\end{proof}}
\definecolor{gris}{gray}{0.45}
\setlength{\parindent}{1cm}
\newcommand{\textcalli}[1]{{\small{\textbf{$\negmedspace$\calligra #1}}}}

\renewcommand{\chaptermark}[1]{\markright{\thechapter\ #1}}
%\renewcommand{\sectionmark}[1]{\markright{\thesection\ #1}}
\fancyhf{} % supprime les en-têtes et pieds prédéfinis
\fancyhead[R]{\thepage}% Left Even, Right Odd
\fancyhead[L]{\textsl{\leftmark}} % Left Odd
%\fancyhead[RE]{\textsl{\leftmark}} % Right Even
\renewcommand{\headrulewidth}{0pt}% filet en haut de page
\renewcommand{\footrulewidth}{0pt} % pas de filet en bas
\fancypagestyle{plain}{ % pages de tetes de chapitre
\fancyhead{} % supprime l’entete
\fancyhead[R]{\thepage}
\renewcommand{\headrulewidth}{0pt} % et le filet
}

\begin{document}
\includepdf{pageGarde.pdf}
%newpage
%\thispagestyle{empty}
%\null
%\newpage
\pagenumbering{roman}
\chapter*{Remerciements}
\renewcommand{\leftmark}{REMERCIEMENTS}
%\addcontentsline{toc}{chapter}{Remerciements}

Nous remercions ...\\

\newpage
\renewcommand{\leftmark}{TABLE DES MATI\`{E}RES}
\thispagestyle{fancy}
\tableofcontents


\newpage
\pagenumbering{arabic}
\renewcommand{\leftmark}{INTRODUCTION}
\chapter{Introduction}
\subsubsection*{Contexte}
\subsubsection*{Définition du problème}
Le probleme dumodel-checking consiste a vérifier qu’un système informatique satisfait une spécification
quand ceux-ci sont donnes sous la forme de modèles mathématiques. Des spècifications typiques sont : est-
ce que le systeme peut atteindre unétat de deadlock ? Est-ce qu’une requète reçoit toujours une reponse ?
Les modeles utilisés peuvent varier : les comportements du système informatique peuventétre modelisè
par un automate acceptant des mots infinis, tandis que la specification peutètre modelisèe par une formule
de logique temporelle LTL.
Plutot que de verifier qu’un système informatique satisfait une spécification, on peut aller plus loin en
envisageant la synthese de controleur. Dans le but de definir les interactions d’un système informatique
avec son environnement, on considere ici un graphe orienté dont les sommets sont partagès entre le système
et l’environnement. Une interaction est alors un chemin infini dans le graphe tel qu’en tout sommet 
systeme (resp. de l’environnement), c’est lui qui décide quel arc suivrea partir de ce sommet. L’objectif
du systeme est, par exemple, d’éviter unètat de deadlock quoique fasse l’environnement, ou encore de
repondrea une requéte quoique fasse l’environnement. Pour y arriver, il a besoin d’une strategie gagnante
qui n’est rien d’autre qu’un programme de controle. La synthese de controleur revient donca construire
(quand c’est possible) une strategie gagnante (contre l’environnement) pour un objectif donnè du système.
Dans ce projet, on propose d’etudier ce problème de synthése pour desjeux de surete´ joues sur graphes.
Pour ces jeux, le systeme a pour objectif d’éviter de passer par certains sommets du graphe.  
\subsubsection*{Présentation et limitations des solutions existantes}
Quand le
graphe est fini, il existe des algorithmes simples qui indiquent si le systeme peut y parvenir et qui dans ce cas
indiquent comment jouer (voir par exemple le livre [1]). Quand le graphe est infini, l’article [2] propose 
algorithme partiel qui utilise des SAT solveurs.
\subsubsection*{Objectif du travail et idées principales}
Dans le cadre du projet, l’etudiant sera amenè´ a comprendre
ces algorithmes et a reproduire les expérimentations de l’article [2], et d’envisager une implèmentation de
calcul de strategie gracea la structure de données “binary decision diagrams” [3,4]
\subsubsection*{Brève description du contenu, chapitre par chapitre}
\newpage


%ICI COMMENCE LE CHAPITRE 2
\chapter{Jeux joués sur graphes}\label{ch:1}
\renewcommand{\leftmark}{CHAPITRE \thechapter.~~Jeux joués sur graphes}
\section*{Jeux de sûreté}
Un \emph{jeu de sûreté} se joue sur une \emph{arène} $A = (V_0,V_1,E)$ composée de deux ensembles disjoints, non-vides de sommets $V_0$ et $V_1$, avec $V_0 \cup V_1 = V$, et d'un ensemble d'arêtes $E \subseteq V\times V$.\\
Un tel jeu est défini par un triplet $\mathfrak{G} = (A, I, F)$, avec $A$ une arène, $I\subseteq F$ un ensemble de sommets initiaux et $F\subseteq V$ un ensemble de sommets $safe$. \\
Le jeu est joué par deux \emph{joueurs} numérotés $j_0$ et $j_1$ qui déplacent tour à tour un pion le long des arêtes de l'arène. Une $partie$ d'un jeu est une séquence infinie $\pi=v_0 v_1 ...$ où $v_0\in I, \forall i\in \mathbb{N},v_i \in V, $ et $(v_i, v_{i+1})\in E$. Une partie est gagnante pour le joueur $j_0$ si $\forall i \in \mathbb{N}, v_i \in F$, sinon le joueur $j_1$ l'emporte.\\
Les coups des joueurs sont décidés par la $strat\text{é}gie$ adoptée par ces derniers. Une stratégie est une fonction $s_{j_n} : V^*V_n\to V$ qui indique vers quel sommet déplacer le pion selon la séquence de déplacement précédents.
Une stratégie peut être \emph{sans mémoire} $s_{j_n} : V_n \to V$ si elle ne prend en compte que le sommet actuel où se trouve le pion.\\
Calculer une stratégie gagnante pour un joueur revient généralement à calculer un ensemble gagnant $W\subseteq V$ pour ce joueur.
$$W_n=\{v\in V|j_n\text{ gagne à partir de }v\}$$
Cette définition d'un ensemble gagnant donne immédiatement une stratégie pour le joueur $j_0$. A chaque tour, $j_0$ n'a qu'à bouger le pion vers un sommet dans $W$.
\section*{Jeux d'atteignabilité}
Un \emph{jeu d'atteignabilité} est similaire à un jeu de sûreté $\mathfrak{G} = (A, I, F)$, sauf que $F \subseteq V$ est un ensemble de sommet à atteindre, et donc $I\subseteq V\setminus F$. Une partie $\pi=v_0v_1...$ est donc gagnante pour le joueur $j_0$ si $\exists i \in \mathbb{N}, v_i \in F$


\chapter{Cas fini}
\renewcommand{\leftmark}{CHAPITRE \thechapter.~~Cas fini}
Une \emph{arène finie} est une arène $A=(V_0, V_1, E)$ pour laquelle $V_0$ et $V_1$ sont des ensembles finis.
\section*{Résolution via les attracteurs}
Une méthode pour calculer les ensembles gagnants des jeux d'atteignabilité se base sur le principe d'attracteur.\\ Soit un jeu d'atteignabilité $\mathfrak{G} = (A, I, F)$ avec $A=(V_0, V_1, E)$ une arène finie, on a :\\
$Attr_{j_n}^i=\{v\in V|\text{ le joueur }j_n\text{ peut forcer une visite d'un sommet de }F\text{ depuis }v\text{ en } \leq  i \text{ déplacements}\}$\\
Construction par induction :
%%% INDUCTION ATTRACTEUR
\begin{equation}
\begin{split}
Attr^0_{j_0}(F)&=F\\
Attr^{i+1}_{j_0}(F)&=Attr^i_{j_0}(F)\\
			&\cup \{v'\in V_0|\exists(v,v')\in E:v\in Attr^i_{j_0}(F)\}\\
			&\cup \{v'\in V_1|\forall(v,v')\in E:v\in Attr^i_{j_0}(F)\}
\end{split}
\end{equation}
On obtient une séquence d'attracteurs $Attr_{j_0}^0(F)\subseteq Attr_{j_0}^1(F) \subseteq ...$ laquelle sera fixe à partir d'une certaine itération vu que $V$ est un ensemble fini. On notera donc $Attr_{j_0}(F)=\bigcup_{i=0}^{|V|}Attr_{j_0}^i$\\


Cette construction de l'attracteur correspond à l'ensemble gagnant de $j_0$.\\
En effet on a que $W_0 \subseteq Attr_{j_0}$ car
\begin{itemize}
\item $\forall v \in Attr_{j_0}^{i+1}\cap V_0, v$ possède un successeur dans $Attr_{j_0}^i$.
\item $\forall v \in Attr_{j_0}^{i+1}\cap V_1,$ tous les successeurs de $v$ sont dans $Attr_{j_0}^i$.
\item $Attr_{j_0}^0(F) \subseteq F$\\
\end{itemize}
Donc $j_0$ peut gagner la partie à partir de tous les sommets de $W_0$.\\
Pour montrer que $Attr_{j_0}(F)\subseteq W_0$, il faut montrer que $j_0$ ne peut pas gagner la partie à partir d'un sommet hors de $Attr_{j_0}(F)$, autrement dit que $j_1$ peut forcer le pion à rester en dehors de $Attr_{j_0}(F)$ depuis tout sommet hors de $Attr_{j_0}(F)$.\\
Soit un sommet $v\in V_1\setminus Attr_{j_0}(F)$, alors $v$ possède au moins une arête $(v, v')$ avec $v'\notin Attr_{j_0}(F)$, sinon on aurait $v \in Attr_{j_0}(F)$.\\
Soit  un sommet $v\in V_0\setminus Attr_{j_0}(F)$, alors toutes les arêtes $v$
mènent vers un sommet hors de $Attr_{j_0}(F)$, sinon on aurait $v \in Attr_{j_0}(F)$.\\
Ces deux cas étant exhaustifs, on a bien que $Attr_{j_0}(F)\subseteq W_0$ et donc $W_0 = Attr_{j_0}(F)$.\\


$Attr_{j_0}(F)$ étant l'ensemble gagnant pour $j_0$, il lui permet d'établir sa stratégie. A chaque tour, le joueur va déplacer le pion d'un sommet de $Attr_{j_0}^{i+1}(F)$ vers un sommet de $Attr_{j_0}^i(F)$ jusqu'à atteindre $Attr_{j_0}^0(F) = F$\\



%%% ALGO ATTRACTEURS
\begin{algorithm}
\caption{Attracteur}\label{attractor}
\hspace*{\algorithmicindent} 
\begin{tabular}{lll}
	\textbf{Entrées} & \textbf{G} : &Graphe, structure de donnée composée d'un tableau à deux\\
	&&dimensions $predecessors$ de prédecesseurs (la liste de \\
	&&prédecesseurs d'un noeud $i$ est stockée à la \\
	&&$i^e$ entrée du tableau) et une liste $players$ (le $i^e$ noeud\\
	&&appartient au joueur dont le numéro figure en \\
	&&$i^e$ entrée de $players$).\\
	&\textbf{F} : &Liste de numéro de sommets.\\
	&\textbf{p} : &Numéro de joueur.\\
	&\textbf{i} : &Nombre d'itération pour la construction de\\
	&&l'attracteur, une valeur négative calculera l'attracteur\\
	&&complet.\\
	\textbf{Sortie} &\multicolumn{2}{l}{$Attr_p^i(F)$}\\
\end{tabular}\\
\begin{algorithmic}[1]
\Procedure{Attractor}{$G, F, p, i$}
	\State $out\_degrees \gets$ tableau de taille $|G|$\Comment Pré-traitement
	\For{$j$ allant de 0 à $|G|-1$}
		\If{$G.players[j] \neq p$}
			\State $out\_degrees[j] \gets $ demi-degré extérieur du noeud $j$
		\EndIf
	\EndFor
	
	\State $attractor \gets$ tableau de taille $|G|$\Comment Initialisation
	\For{$index$ in $F$}
		\State $attractor[index] \gets 1$
	\EndFor
	\State $attractor\_new \gets F$
	
	\While{$attractor\_new$ non-vide \textbf{and} $i\neq 0$}\Comment Calcul de l'attracteur
		\State $to\_check \gets attractor\_new$
		\State $attractor\_new \gets [$ $]$
		\For{$index$ in $to\_check$}
			\For{$pred$ in $G.predecessors[index]$}
				\If{$attractor[pred] = 0$}
					\If{$G.players[pred] = p$}
						\State $attractor\_new.append(pred)$
					\Else
						\State $out\_degrees[pred]\gets out\_degrees[pred]-1$
						\If{$out\_degrees[pred]=0$}
							\State $attractor\_new.append(pred)$
						\EndIf
					\EndIf
				\EndIf
			\EndFor
		\EndFor
	\algstore{attractor}
\end{algorithmic}
\end{algorithm}
\newpage
\begin{algorithm}
\begin{algorithmic}[1]
	\algrestore{attractor}
		\For{$index$ in $attractor\_new$}
			\State $attractor[index]\gets 1$
		\EndFor
		\State $i\gets i-1$
	\EndWhile
	\State\Return $attractor$
\EndProcedure
\end{algorithmic}
\end{algorithm}

\chapter{Cas infini}

\chapter*{Conclusion}
\addcontentsline{toc}{chapter}{Conclusion}
\renewcommand{\leftmark}{CONCLUSION}

Mettez votre conclusion ici.  Dressez le bilan de votre travail effectué, en prenant du recul. Discuter de si vous avez bien réussi les objectifs du travail ou non. Présentez les perspectives futurs.


%Le style bibliographique utilisŽ
\bibliographystyle{latex8}

%Le fichier .bib uitilisŽ
\bibliography{biblio}

\newpage
\appendix
\addcontentsline{toc}{chapter}{Annexes}

\chapter{Premi\`ere annexe}
\renewcommand{\leftmark}{ANNEXE \thechapter.~~Premi\`ere annexe}
\label{annexe1}

\chapter{Deuxi\`eme annexe}
\renewcommand{\leftmark}{ANNEXE \thechapter.~~Deuxi\`eme annexe}
\label{annexe2}

%%%%%%%FIN-ANNEXES%%%%%%%%%%
\end{document}
